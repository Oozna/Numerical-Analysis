\documentclass[a4paper]{article}

\usepackage{amssymb, amsmath}
\usepackage{listings}
\usepackage{xcolor}
\usepackage[margin=1in]{geometry}

\renewcommand{\arraystretch}{1.5}

\definecolor{codegreen}{rgb}{0,0.6,0}
\definecolor{codegray}{rgb}{0.5,0.5,0.5}
\definecolor{codepurple}{rgb}{0.58,0,0.82}
\definecolor{backcolour}{rgb}{0.95,0.95,0.92}

\lstdefinestyle{mystyle}{
    backgroundcolor=\color{backcolour},   
    commentstyle=\color{codegreen},
    keywordstyle=\color{magenta},
    numberstyle=\tiny\color{codegray},
    stringstyle=\color{codepurple},
    basicstyle=\ttfamily\footnotesize,
    breakatwhitespace=false,         
    breaklines=true,                 
    captionpos=b,                    
    keepspaces=true,                 
    numbers=left,                    
    numbersep=5pt,                  
    showspaces=false,                
    showstringspaces=false,
    showtabs=false,                  
    tabsize=4
}

\lstset{style=mystyle}




\author{Eric Johansson}
\title{Översättningsblad från matlab till Python}

\begin{document}
\maketitle
Detta översättningslexikon förväntar sig att numpy har blivit 
importerat som `np', sympy som `sy' och pyplot från matplotlib som `plt'. 

Exempel enligt detta 
\begin{lstlisting}[language=Python]
  import numpy as np
  import sympy as sy
  from matplotlib import pyplot as plt
\end{lstlisting}

\begin{center}
\begin{tabular}{|c | c | c|}
    \hline
    & Matlab & Python \\
    \hline
    Skapa vektor mellan 0 och 10 med steglängden 0.1 & x = [0;10;0.1] & x = np.arange(0,10,0.1) \\
    \hline
        for loop från 1 till 15                      & for i in 1:15 & for i in range(1,15): \\
    \hline
        Skriva ut till skärmen & disp("Bla bla") & print("Bla bla") \\
    \hline

\end{tabular}
\end{center}

\end{document}
