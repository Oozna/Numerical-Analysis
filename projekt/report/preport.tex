\documentclass{article}

\usepackage{listings}

\title{Python for numerical analysis}

\author{Eric Johansson\\Can Kupeli\\Samuel Greenberg}



\begin{document}
\maketitle
\newpage



\section{Introduction}%
\label{sec:introduction}
Throughout the years there have been many different languages used for mathematical computing.
Traditionally the base algorithms, functions and libraries are created in a low level language like C, C++ or Fortran and then wrapped
in a higher order language such as Matlab, Julia or Python.

This report aims to analyse the use of Python for numerical computation and compare it to Matlab.

\section{Background}%
\label{sec:background}

Python was first created in the late 1980s by Guido van Rossum in the Netherlands and has since gone through three different stages.
It is a general purpose programming language (GPL) which is open source and has a very thriving community. This is the thing that has
directly led to the feature that allows Python to be an effective language for mathematical purposes, namely NumPy.
NumPy is a library for Python which uses arrays as their base data structure which allows for fast and efficient handling of data.
SciPy is another library made by the same author which includes many solvers and operations which are used in the scientific field, such
as regression formulas, integral solvers, ordenary differential equation solvers and more.
Another vital add-on to the Python ecosystem is MatPlotLib, which allows users to visualize their data in many different ways.
An alternative to Matlabs symbolic toolbox can be found in the SymPy library. It introduces a way to work symbolically to simplify
equations, find derivatives, antiderivatives and much more. When referencing to Python in the rest of the paper it is assumed that these
libraries are included.

Matlab on the other hand is a program which was also released in the 1980s. It's developed and owned by MathWorks in Massachussets US and
is as such a proprietary software. It was originally not even a programming language but a ``simple'' interactive matrix calculator which
is why it uses a matrix as its default type for values. The underlying language was initially written in Fortran but later rewritten
in C. However unlike C, Matlab is not a GPL but is specifically meant to be used for mathematical calculations. Matlab is as mentioned before a proprietary software that is distributed through a pay for license model with yearly subscriptions.

\section{Python compared to Matlab}%
\label{sec:pcm}
Recently there has been a trend for mathematicians and scientists to move away from Matlab and transition towards Python and other
languages such as Julia. There can be many different reasons for this and each persons decision is unique to them, however here we will
compare some of the reasons as to why someone would or would not use one language over the other.

\subsection{Open source}%
Most modern day programming-languages and frameworks are open source which allows users of the product to directly contribute to the
project by implementing new features, fixing bugs, creating optimizations and more. This is not only limited to the core part of the
languages such as compilers and standard functions but it also allows for easier creation of libraries and customizability of the code
they are running.

As mentioned in chapter`~`\ref{sec:background} Python is open source whilst Matlab is not. For some people this is a deal-breaker,
presumably more so in the mathematical and scientific world since the scientific method revolves around sharing information and knowledge
to build great things together. Another reason is the financial applications which creates a lower barrier for entry to start using the
software.

However it may also be comforting to know that the tool you are using is backed and maintained by a large corporation and only
professional developer which is the case for Matlab.

\subsection{Performance}%
The performance between the two languages are quite comparable, while there isn't much scientific litterature on this topic anecdotally
they compare quite similarly when adhering to the ways the different languages are intended to be used. For example passing by value for
python instead of passing by index. This is not very surprising since both languages use C as their underlying ``engine'' and uses Basic
Linear Algebra Subprograms (BLAS) as the low-level routines for performing common linear algebra operations.

\subsection{Uses outside of maths}
Python, being a GPL, is used for many things outside of mathematical computation. Python is very common in the Machine Learining and Data
analysis fields, web-frameworks such as Flask and Django power some of the biggest sites on the internet, such as Youtube, and Python is
also useful as a general scripting language for automations, webscrapers et.c. Matlabs killer feature in this domain is the second
biggest product made by Mathworks namely Simulink. It's used to simulate and analyze multidomain dynamical systems with ``graphical block diagramming tool''-interface.

\subsection{Style}
Since Python is meant for much more than just maths it requires libraries to be included and referenced to use it to the fullest extent.
This creates some rather annoying and ugly syntax like writing:
\begin{lstlisting}[language=Python]
y = lambda x: math.exp(x*math.sin(2*x))
\end{lstlisting}
needing to reference math.\ every time while in Matlab the code looks more concise.
\begin{lstlisting}[language=Matlab]
y = @(x) exp(x*sin(2*x))
\end{lstlisting}
This is a result of Pythons need to not hog up the standard namespace since such operations may not be necessary for someone creating a
web application. Matlab has it's clear purpose and can therefore afford to make it simple for it's developers.



\section{Python style}%
\subsection{Formatting}
Python has, unlike many other languages, a set standard for how it's supposed to be written. These are defined in the pep8 guidelines
which states, among other things, wether you should use spaces or tabs, snakecase or camelcase and much more. However this is not
enforced by the language so users are free to do as they please but it creates a good goal for how readable Python code is supposed to
be written.

As such mathematicians have the possibility of writing their formulas to closely resemble the equations or functions they are solving
by using inline (lambda) functions, even though it is not recommended by the language to do so. Readability in programming and math are
not always the same where programmers often prefer to keep things modular, separated and humanly readable mathematicians often prefer
to write in such a way that it's as similar to the way they would write on paper, few parentheses, assigning functions to variables et.c.
Python enables people to do that, but strongly encourage them not to.

\subsection{Programming paradigm}
When it comes to vector or matrix algebra it's recommended to use the built in functions of NumPy and to vectorize the problems
as much as possible to avoid loops.

If we have an array with values 1,2,\ldots,100 and would like to increment each by one a new programmer or someone coming from another
language may create a loop like this:
\begin{lstlisting}[language=Python]
for element in array:
  element += 1
\end{lstlisting}
However both the performance and readability of this is much lower compared to the recommended method (assuming that the array is a
numPy array):
\begin{lstlisting}[language=Python]
array += 1
\end{lstlisting}

As mentioned this comes with huge performane gains since numPy is built in C and has internal optimizations for matrix algebra and also
expresses intent more whilst taking up less space.

There are also other features that makes Python avoid loops such as list comprehension, however those methods are not as applicable
in the mathematical field as they are in software development and as such will not be discussed here




\section{Conclusion}%
While there is no real conclusion to the question if one should use Python for mathematical computing or not it is good to know that
it is a serious and growing option in the field. Many schools such as KTH have already replaced Matlab with Python for some of their
engineering programmes. The style may not be optimal and can look cluttered with a lot of namespace references but the familiarity with
it can be used to transition into other fields and aid in become both a better programmer and mathematician at the same time.




\end{document}
