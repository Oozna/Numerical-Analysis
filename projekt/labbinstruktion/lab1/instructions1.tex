\documentclass[a4paper]{article}
\usepackage{amsmath, amssymb}
\usepackage[swedish]{babel}
\usepackage{graphicx}
\graphicspath{{imgs/}}
\usepackage{float}
\usepackage[hidelinks]{hyperref}
\usepackage{fancyhdr}
\usepackage[parfill]{parskip}
\usepackage[margin=1in]{geometry}
\usepackage{listings}
\usepackage{xcolor}
\usepackage[style=ieee, backend=biber]{biblatex}
\usepackage{csquotes}
\addbibresource{../sources.bib}
\setcounter{secnumdepth}{0}
\pagestyle{fancy}
\fancyhf{}
\setlength{\headheight}{12.31253pt}
\newcommand{\getauthor}{Eric Johansson (erjo2002@student.miun.se)} %Author
\newcommand{\gettitle}{Laboration 1 \\ Ekvationer och differensformler} %Title
\newcommand{\getcourse}{(MA069G, Matematisk Modellering, 6hp)} %Course
\newcommand{\getsupervisor}{Magnus Eriksson}


\definecolor{codegreen}{rgb}{0,0.6,0}
\definecolor{codegray}{rgb}{0.5,0.5,0.5}
\definecolor{codepurple}{rgb}{0.58,0,0.82}
\definecolor{backcolour}{rgb}{0.95,0.95,0.92}

\lstdefinestyle{mystyle}{
    backgroundcolor=\color{backcolour},   
    commentstyle=\color{codegreen},
    keywordstyle=\color{magenta},
    numberstyle=\tiny\color{codegray},
    stringstyle=\color{codepurple},
    basicstyle=\ttfamily\footnotesize,
    breakatwhitespace=false,         
    breaklines=true,                 
    captionpos=b,                    
    keepspaces=true,                 
    numbers=left,                    
    numbersep=5pt,                  
    showspaces=false,                
    showstringspaces=false,
    showtabs=false,                  
    tabsize=2
}

\lstset{style=mystyle}

\begin{document}

\begin{titlepage}
  \begin{center}
    \vspace*{1cm}
    \includegraphics[width=0.8\textwidth]{msu.png}\\[0.5cm]
    \Large
    Institutionen för matematik och ämnesdidaktik (MOD)\\[1cm]
    \Huge
    \textbf{\gettitle}

    \large
    \getcourse{}

    \vspace{1cm}

    \vfill
    \vspace{0.8cm}
    \small
    \today \\
    \Large
    \textbf{Handledare:}\\
    \getsupervisor{}
  \end{center}
\end{titlepage}

\lhead{Laboration 1}
\lhead{Ekvationer och differensformler}
\tableofcontents
\newpage

\section{Förord}
Denna laboration ger er möjlighet att bekanta er med hur man använder
Python för att lösa några enkla ickelinjära ekvationer samt undersöka
några finita differensformler.

Det rekommenderas att ni använder er av verktyget Jupyter Notebooks för att lösa
uppgifterna. I det verktyget kan man sammanställa block med Python-kod, beräkningsresultat,
plottar och egen text med ekvationer i ett och samma dokument, och köra all kod i dokumentet
i följd eller interaktivt. Jyputer Notebook ingår i utvecklingsmiljön Anaconda, som du med
fördel kan välja att installera och använda, eftersom den innehåller Pythonbibliotek och
verktyg som är vanliga inom beräkningsvetenskap, datamining och maskininlärning, och som passar ihop.

För att utföra laborationen rekommenderas det att biblioteken numPy \cite{numpy-main},
symPy \cite{sympy-main} och matplotlib \cite{matplotlib-main} används. De är inkluderade i Anaconda men kan annars installeras genom att
skriva ``pip install \textit{namn}'' i terminalen i Windows eller ``pip3 install \textit{namn}'' på Mac eller Linux. 
För många vanliga matematiska funktioner som till
exempel sinus, kvadratroten och log kan det krävas att ni importerar math-biblioteket i er skript men generellt kan numPys
verisioner av dessa funktioner användas istället.

\textbf{Instruktionerna kommer utgå ifrån att ni har skrivit detta i början av er fil / notebook}
\begin{lstlisting}[language=Python]
import math 
import numpy as np
import sympy as sy
from matplotlib import pyplot as plt
\end{lstlisting}

Därefter kan ni använda funktionaliteten från de olika biblioteken genom att
kalla deras funktioner. Exempelvis math.sqrt(), np.cos(), sy.symbos()

I rapporten skall ni bifoga er kod, beräkningsresultat, diagram, korta svar och förklaringar.
Allas namn och datum måste ingå i dokumentet.

\newpage
\section{Uppgift 1}

Betrakta funktionen \( f(x) \) enligt:
\begin{equation} \label{eqn:fxq1}
  f(x)=\frac{\sqrt{(x-5)^5}+2\cos (\pi\sqrt{x})}{\sqrt{x+4 \ln (x-\pi)}-1}
\end{equation}
Om vi vill beräkna \( f'(x) \) i en viss punkt \( x=x_0 \)
(i detta fall är \( x_0=7 \)) kan vi använda differensformlerna

\begin{equation} \label{eqn:framdiff}
  f'(x)=\frac{f(x+h)-f(x)}{h}+E_1(h)=D_1(x,h)+E_1(h)
\end{equation}
\begin{equation} \label{eqn:centerdiff}
  f'(x)=\frac{f(x+h)-f(x-h)}{2h}+E_2(h)=D_2(x,h)+E_2(h)
\end{equation}

där (\ref{eqn:framdiff}) är framåtdifferens och (\ref{eqn:centerdiff}) är centraldifferens.

Skriv ett skript som berkänar och skriver ut \( D_1(x_0,h) \)
och  \( D_2(x_0,h) \) för \( h=0.04,0.02,0,01 \).

Ett mycket noggrant resultat som är korrekt till 15 decimaler är

\begin{equation} \label{eqn:fxexakt}
  f'(x_0)\approx 2.164821222303575
\end{equation}

\textbf{Svara på följande frågor i rapporten}\\
Vilken av formlerna  \( D_1(x_0,h) \) och  \( D_2(x_0,h) \) var noggrannast,
d.v.s skilde sig minst från värdet i (\ref{eqn:fxexakt})? Varför är den formeln mer noggrann?


\section{Uppgift 2}
Om man skulle testa att manuellt bestämma \( f'(x) \) av (\ref{eqn:fxq1})
skulle man inse att det blir rätt så bökigt. Däremot kan man använda sig av Sympy för att räkna det symboliskt.

Skriv först
\begin{lstlisting}[language=Python]
import sympy as sy

x = sy.symbols("x")
f = (sy.sqrt((x-5)**5) + 2*sy.cos(sy.pi*sy.sqrt(x)))/(sy.sqrt(x+4*sy.log(x-sy.pi))-1)
\end{lstlisting}
vilket anger \( f(x) \) symboliskt i Python. Därefter skriv:

\begin{lstlisting}[language=Python]
  f_prim = f.diff()
\end{lstlisting}
vilket ger uttrycket för derivatan.

Sedan, skriv

\begin{lstlisting}[language=Python]
  f_prim_x0=f_prim.subs(x,7)
\end{lstlisting}
Detta byter ut \(x\)  till 7. För att sedan se resultatet skriv:

\begin{lstlisting}[language=Python]
  print(f_prim_x0.evalf(16))
\end{lstlisting}
Detta kommer skriva ut svaret med 16 decimaler och ni kan se
att det (i princip) exakta värdet på derivatan är 2.164821222303575


\textbf{Svara på följande frågor i rapporten}\\
Vad betyder det att räkna symboliskt?
Vad kallas det när vi räknar som vanligt som i Uppgift 1?

\newpage
\section{Uppgift 3}

Med hjälp av Taylorutveckling kan man visa att framåtdifferensen i (\ref{eqn:framdiff}) har
en trunkeringsserie \( E_1(h)\approx k_1h(O(h)) \)(jämfört med centralformeln i (\ref{eqn:centerdiff})
som har \(E_1(h)\approx k_1h_2(O(h^2))\), där konstanterna \(k_1\)  och \(c_1\)  är oberoende av \(h\).

Om vi då tar framåtdifferensen i (\ref{eqn:framdiff}) får vi:
\begin{equation} \label{eqn:framdiff_k}
  f'(x_0)\approx \frac{f(x_0+h)-f(x_0)}{h}+k_1h
\end{equation}

Om vi istället använder halva stegkängden, \( \frac{h}{2} \) , får vi:
\begin{equation} \label{eqn:framdiff_halva_h}
  f'(x_0)\approx \frac{f(x_0+\frac{h}{2})-f(x_0)}{\frac{h}{2}}+k_1 \frac{h}{2}
\end{equation}

Differensen mellan (\ref{eqn:framdiff_k}) och (\ref{eqn:framdiff_halva_h}) ger att vi kan bestämma \( k_1 \) till:

\begin{equation} \label{eqn:k}
  k_1=\frac{2\left(2f\left(x_0+\frac{h}{2}\right)-f\left(x_0+h\right)-f\left(x_0\right)\right)}{h^2}
\end{equation}

Använd \( h=0.02 \) i (\ref{eqn:k}) och bestäm ett värde på \( k_1 \). Använd sedan värdet på \( k_1 \) i (\ref{eqn:framdiff_k})
för att bestämma ett värde på \( f'(x_0) \)

\textbf{Svara på följande frågor i rapporten}
Varför blir inte resultatet exakt fastän vi beräknat ut \( k_1 \)?

\section{Uppgift 4}
Det finns även differensformler för andraderivatan: (Centraldifferensen för andraderivatan)
\begin{equation} \label{eqn:fbisscenter}
  f''(x)=\frac{f(x-h)-2f(x)+f(x+h)}{h^2}+E_3(h)=D_3(x,h)+E_3(h)
\end{equation}

Beräkna andraderivatan med (\ref{eqn:fbisscenter}) i punkten \( x_0=7 \)  med samma värden på
\( h  \)  som i Uppgift 1 och jämför med det exakta värdet som kan beräknas på samma sätt som
i Uppgift 2 fast vi måste lägga till

\begin{lstlisting}[language=Python]
f_biss = f_prim.diff()
f_biss_x_0 = f_biss.subs(x,7)
print(f_biss_x_0.evalf(16))
\end{lstlisting}

Exakta värdet ska bli 1.806906812550099. Använd även samma värde på \(h\)  som i Uppgift 3
och extrapolera.

\section{Uppgift 5}
Som vi sett i de tidigare uppgifterna tenderar ett lägre \(h\) att ge en approximering som är närmre det exakta värdet.
I denna uppgift ska vi se hur litet vi kan göra \(h\). I Python finns ett tal, \(\epsilon\), som är det minsta talet så att \(1+\epsilon>1\). Detta tal kan du ta reda på genom att skriva:

\begin{lstlisting}[language=Python]
  import sys
  print(sys.float_info.epsilon)
\end{lstlisting}

Testa approximera derivatan av \(f\) med hjälp av (\ref{eqn:framdiff}) med succesivt lägre värde på \(h\) från 0.1
till \(\epsilon\).

\textbf{Svara på följade frågor i rapporten}

Vad händer med approximationen när \(h\) blir lägre och lägre? Varför?


\newpage
\section{Uppgift 6}
Nu ska vi ta reda på samtliga nollställen till funktionen:
\begin{equation} \label{eqn:F}
  F(x)=-\left(\frac{x}{5}+\cos x\right)
\end{equation}

Börja med att plotta kurvan genom att skriva:
\begin{lstlisting}
    x = np.arange(-100,100,0.1)
    F = -(x/5+np.cos(x))

    plt.title("Uppgift 5") 
    plt.xlabel("x") 
    plt.ylabel("F(x)") 
    plt.xlim(a,b) 
    plt.grid()
    plt.axhline(linewidth = 1, color = "black")
    plt.axvline(linewidth = 1, color = "black")
    plt.plot(x,F)
    plt.show()
\end{lstlisting}
Ändra \( a  \) och \( b  \) så att ni kan se ungefär var alla nollställen ligger.


\section{Uppgift 7}
Problemet i (\ref{eqn:F}) kan lösas med Newton-Raphsons metod:
\begin{equation} \label{eqn:newton_rhapson}
  x_{n+1}=x_n-\frac{F(x_n)}{F'(x_n)}
\end{equation}


\textbf{Svara på följande frågor i rapporten}\\
Algoritmen i (\ref{eqn:newton_rhapson}) konvergerar inte alltid. När blir det problem i (\ref{eqn:newton_rhapson})?
Vad händer då man startar med \( x_0 = 3  \)? Bestäm samtliga nollställen
till (\ref{eqn:F}) med hjälp av (\ref{eqn:newton_rhapson}) och anteckna antalet
iterationer. Testa med 6 st olika toleranser för lösningen.


\section{Uppgift 8}
Problemet i (\ref{eqn:F}) kan även lösas med sekantmetoden:
\begin{equation} \label{eqn:sekantmetoden}
  x_{n+2}=x_{n+1}-\frac{(x_{n+1}-x_n)F(x_{n+1})}{F(x_{n+1})-F(x_n)}
\end{equation}


Bestäm samtliga nollställen till (\ref{eqn:F}) med hjälp av
(\ref{eqn:sekantmetoden}) och anteckna antalet iterationer.
Testa med 6 st olika toleranser för lösningen.



\textbf{Svara på följande frågor i rapporten}\\
Fundera vilka fördelar och nackdelar som finns med de olika metoderna
(\ref{eqn:newton_rhapson}) och (\ref{eqn:sekantmetoden}) och blev det
någon skillnad i antalet iterationer?


\section{Uppgift 9}
I den sista uppgiften ska vi lösa problemet i (\ref{eqn:F}) med Sympys ``nsolve'':

\begin{lstlisting}[language=Python]
x = sy.Symbol('x')
print(sy.nsolve(-x/5-sy.cos(x)), start))
\end{lstlisting}

Byt ut `start' mot lämpliga startvärden och bestäm samtliga nollställen.












\end{document}
